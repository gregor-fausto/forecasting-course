\documentclass[12pt, oneside]{article}   	% use "amsart" instead of "article" for AMSLaTeX format

%%%%%%%%%%%%%%%%%%%%%%%%%%%%%%%%%%%%%%%%%%%%%%%%%%
% set up packages, geometry
%%%%%%%%%%%%%%%%%%%%%%%%%%%%%%%%%%%%%%%%%%%%%%%%%%
\usepackage{geometry}
\usepackage{textcomp}
\usepackage{amsmath}
\usepackage{graphicx}
\usepackage{amssymb}
\usepackage{fancyhdr}
\usepackage{subcaption}
\usepackage{bm}
\usepackage[superscript,noadjust]{cite} % puts dash in citations to abbreviate
	
\geometry{letterpaper, marginparwidth=60pt}                   		

%\usepackage [autostyle, english = american]{csquotes} % sets US-style quotes
%\MakeOuterQuote{"} % sets quote style

%%%%%%%%%%%%%%%%%%%%%%%%%%%%%%%%%%%%%%%%%%%%%%%%%%
% set up code
%%%%%%%%%%%%%%%%%%%%%%%%%%%%%%%%%%%%%%%%%%%%%%%%%%
\usepackage{hyperref}
\hypersetup{
    colorlinks=true,
    linkcolor=blue,
    filecolor=magenta,      
    urlcolor=cyan,
}

\usepackage{etoolbox}
\AtBeginEnvironment{quote}{\small}

\usepackage{float}
\usepackage{color}

\usepackage{pgf, tikz, eqnarray}
\usetikzlibrary{arrows, automata}

%%%%%%%%%%%%%%%%%%%%%%%%%%%%%%%%%%%%%%%%%%%%%%%%%%%%
\pagestyle{plain}                                                      %%
%%%%%%%%%% EXAFT 1in MARGINS %%%%%%%                                   %%
\setlength{\textwidth}{6.5in}     %%                                   %%
\setlength{\oddsidemargin}{0in}   %% (It is recommended that you       %%
\setlength{\evensidemargin}{0in}  %%  not change these parameters,     %%
\setlength{\textheight}{8.5in}    %%  at the risk of having your       %%
\setlength{\topmargin}{0in}       %%  proposal dismissed on the basis  %%
\setlength{\headheight}{0in}      %%  of incorrect formatting!!!)      %%
\setlength{\headsep}{0in}         %%                                   %%
\setlength{\footskip}{.5in}       %%                                   %%
%%%%%%%%%%%%%%%%%%%%%%%%%%%%%%%%%%%%                                   %%		

%%%%%%%%%%%%%
% DEFINE CODE BLOCK
%%%%%%%%%%%%%
\usepackage{listings}

\definecolor{dkgreen}{rgb}{0,0.6,0}
\definecolor{gray}{rgb}{0.5,0.5,0.5}
\definecolor{mauve}{rgb}{0.58,0,0.82}

\lstset{frame=tb,
  language=R,
  aboveskip=3mm,
  belowskip=3mm,
  showstringspaces=false,
  columns=flexible,
  basicstyle={\small\ttfamily},
  numbers=none,
  numberstyle=\tiny\color{gray},
 % keywordstyle=\color{blue},
  commentstyle=\color{dkgreen},
  stringstyle=\color{mauve},
  breaklines=true,
  breakatwhitespace=true,
  tabsize=3,
  otherkeywords={0,1,2,3,4,5,6,7,8,9},
  deletekeywords={data,frame,length,as,character,dunif,ps},
}

%%% BIBLIOGRAPHY
\usepackage{natbib}
%\bibliographystyle{abbrvnat}
\setcitestyle{authoryear,open={(},close={)}}

%%% START DOCUMENT

\begin{document} 

\tableofcontents

\clearpage

\section*{General notes}

\href{https://ecoforecast.org/nefi2020/}{Link to Ecological Forecasting Initiative workshop website}.

\section{Characterizing uncertainty}

\noindent Presenter: Shannon LaDeau

Classic assumptions of the linear model:

\begin{itemize}
\item Homoskedasticity
\item No error in X variables
\item Error in Y is measurement error
\item Normally distributed error
\item Pbservations are independent
\item No missing data
\end{itemize}

These assumptions are for data from agricultural experiments, but these assumptions often don't hold for a lot of ecological data. Great for getting a statistically significant slope but that might not be great for making a forecast.

\begin{align}
 y_i \sim \beta_0 + \beta(x_i) + \epsilon_i 
\end{align}

Data model:

\begin{align}
 y_i \sim N(\mu_i, \epsilon_i)
 \end{align}
 
 Process model:
 
 \begin{align}
 \mu_i = \beta_0 + \beta(x_i)
 \end{align}

 Parameter model:
 
 \begin{align}
 \epsilon_i \sim N(0,\tau) \\
 \beta \sim N(\beta_{bar}, \nu)
 \end{align}
 
Can also write this model in graph notation. In this case we break the model into a data model, process model, and parameter model. Three layers where the data model is on top, the process model is second, the parameter model is third. 
 
Beyond the classic assumptions, what are you going to do with real ecological data? What are you actually doing when you're working with data? Assume that the data are random samples from a true population. We describe the population as a distribution because we can't sample everyone so we are using a sample. The distribution let's us make some assumptions about how the sample is related to the whole population. The distribution also helps us estimate what we didn't measure. The expected relationship between the population and the sample is described by a probability distribution.
 
For non-normal data (e.g. a 0/1), we have a data model, a process model that contains a link and a linear model, and then a parameter model. Another assumption that might be violated is constant variance, so could model the variance as a function of data. Shannon presents an example in which the data are modeled with a constant versus a changing variance, which shows that the credible interval changes if using a model with variance changing. Capturing uncertainty in this case does not mean improving your estimate of the response but rather actually means doing a better job of capturing the increase in variance.

Now we move on to observation error. A regression model assumes that all error is in the Y [$Y \sim N(\beta_1 + \beta_2 x_i, \sigma^2)$. But sometimes we also don't measure X very well either, and so there is error in the predictor variables. Imagine that you are getting error in X but you are not measuring that. One way of modeling that is considering errors in variables. Here we are modeling X as a random variable.

 \begin{align}
Y \sim N(\beta_1 + \beta_2 x_i, \sigma^2) \\
X^{(o)} \sim N(\chi, \tau^2)
 \end{align}
 
Now we move on to latent variables. These are any variable that is not directly observed such as: missing data, variables that are measured with error (biased or random), and proxy measures. Ignoring variable latency (e.g. modeling a .... ???)

With a good model, you might be able to make predictions of missing data. For example, you might be able to use regression to fill it in. In a Bayesian framework, the model can predict missing Ys but you can do the same thing and estimate missing predictors. However, to do this you need to develop a model for the missing X and a model for how those missing X influence the Ys. To make this work, you need to assume that data is missing at random. Data that is missing systematically can not be estimated but you need a better model if it's missing systematically. It's a powerful approach but you're not going to do magic with the data. 

An example comes from a forest ecology FACE experiment at a pine plantation in Duke Forest. The experiment measured the seed response to CO2 fumigation. So to get fecundity of trees, there's often a relationship between tree size and seed number (seed number as a function of diameter). In Shannon's data on cone counts, there is a lot of individual variability. So they used data on cones and seeds to inform (latent) fecundity estimates. They were able to combine data from seed baskets to constrain what is observation error and what is natural variability. The study is described in \cite{ladeau2006}.

\section{Hierarchical Bayes}

\noindent Presenter: Mike Dietze

\subsection{Hierarchical models}

Hierarchical models can help us identify as-yet uncharacterized variability. They can help us deal with the fact that we can never deal with all variability.

Imagine that you're confronted with making the same measurement over different observational units. For example, it could be years or sites. One way of dealing with this would be lumping all data together and estimating a single mean ($\mu$ from $Y_1, Y_2, Y_3$) - a common mean. Another would be estimating independent means ($\mu_1$ for $Y_1$, $\mu_2$ for $Y_2$, $\mu_3$ for $Y_3$). In reality, there is often some pattern in between. They provide an intermediate case to represent the continuum where there is some sharing. In a simple hierarchical model for a mean, there would be a mean that you're estimating at each site where you estimate a mean at each site ($\theta$s) and then an across-site mean ($\mu$). 

Partially pooled models allow you to partition the variability within each site, and the variability across sites. Partition into across-site, and within-site. This represents a continuum because it represents what you see in the data. If the data tells you there is a lot of variability from site to site, it will behave more like the independent means model.  If the data tells you there is little variability from site to site, it will behave more like the common means model. The data will tell you how this works.

The model here represents the idea of borrowing strength. The inference that you make at each site borrows strength from all other sites. This is because the across-site mean is the prior for each site mean.

A model with independent means and common variance will look like this:

\begin{align}
 y_i \sim N(\mu_k, \sigma^2) \\
 \mu_k \sim N(\mu, \tau^2) \\
 \sigma^2 \sim IG(s_1, s_2) 
\end{align}

Here, the model is fitting each data set independently but assuming the mean for each has the same prior. 

For a hierarchical model, we want the mean $\mu$ to be an unknown that we're fitting based on the data.

\begin{align}
 y_i \sim N(\mu_k, \sigma^2) \\
 \mu_k \sim N(\mu, \tau^2) \\
 \sigma^2 \sim IG(s_1, s_2) \\
  \mu_k \sim N(\mu_0, V_{\mu}) \\
\tau^2 \sim IG(t_1, t_2)
\end{align}

Here are some of the key take home messages about HM models.

\begin{itemize}
\item We can model variability in the parameters of a model. This is helpful if we are (for example) interested in variability among sites but there is a common process that constrains the process.
\item We can partition variability more explicitly into multiple terms. In the simple example, we partitioned variability in the mean. But this can extend to more complicated cases. 
\item We can borrow strength across data sets. Data-rich parts of analysis can help constrain data-poor parts of the analysis. To the extreme, is when there is no data. Hierarchical models formally let you distinguish between within and out of sample inference. 
\item Details of hierarchical models are usually in the subscripts.
\item Models are typically hierarchical with respect to the parameters. 
\end{itemize}

In relation to the third point, out of sample predictions are possible with HM models.  Formally, we think that out of sample predictions should be more uncertain than in-sample predictions. For example, consider the idea of fitting models to data from 3 sites and then trying to predict for a fourth site. If all sites are fit independently, then we have no way of making predictions about the fourth site.  

\subsection{Random effects}

Random effects are a common \textit{special case} of hierarchical models. Here's the model with partial pooling:

\begin{align}
 y_i \sim N(\mu_k, \sigma^2) \\
 \mu_k \sim N(\mu, \tau^2) \\
 \sigma^2 \sim IG(s_1, s_2) \\
  \mu_k \sim N(\mu_0, V_{\mu}) \\
\tau^2 \sim IG(t_1, t_2)
\end{align}

Here's how the above model is related to random effects

\begin{align}
 y_i \sim N( \mu_g + \alpha_k, \sigma^2) \\
 \alpha_k \sim N(0, \tau^2) \\
 \sigma^2 \sim IG(s_1, s_2) \\
  \mu_g \sim N(\mu_0, V_{\mu}) \\
\tau^2 \sim IG(t_1, t_2)
\end{align}

Some important things to note are that random effects always have a mean 0 (see $\alpha_k$). This is the same as saying that the difference from the global mean is unbiased. Random effects variance attributes a portion of uncertainty to a specific source (subscripted by $k$).

Random effects are generally applied to aspects of a study that would not be the same if replicated. For example, this might include things like plots, block, year, individuals, etc. There is also often some amount of replication for the random effect to be identifiably different from the residual "noise" term $J \sim N(0, \sigma^2) $. In practice, random effects are often used to account for a lack of independence among these units. In contrast, fixed effects include things like treatments and covariates of interest. 

We are using random effects to account for unexplained variance associated with groups. This may point to scales (temporal, spatial) that may need additional explanation. Adding covariates or fixed effects may explain some of the variance but there's always something unmeasured and sometimes additional fixed effects are not justified.

Can use these random effect estimates to look for correlation between variability and covariates without having to rerun the whole model. 

Big points: properly accounting for uncertainty can be just as important for making valid inference from your model as the process model and covariates. Random effects are used to account for the impacts of unmeasured/unmeasurable covariates. 

An example of this approach can be found in \cite{dietze2008}.

\subsubsection{Model 1: Global mean}

\begin{lstlisting}
model{

mu ~ dnorm (0, 0.001)
sigma ~ dgamma(0.001,0.001)

for(t in 1:nt){
	for(b in 1:nb){
		for(i in 1:nrep){
		x[t,b,i] ~ dnorm(mu, sigma)
		}
	}
}
}
\end{lstlisting}

\subsubsection{Model 2: Random temporal effect}

\begin{lstlisting}
model{

mu ~ dnorm (0, 0.001)
sigma ~ dgamma(0.001,0.001)

for(t in 1:nt) { alpha.t[t] ~ dnorm (0, tau.t) }
tau.t ~ dgamma(0.001,0.001)

for(t in 1:nt){
	## process model
	Ex[t] <- mu + alpha.t[t]
	for(b in 1:nb){
		for(i in 1:nrep){
		x[t,b,i] ~ dnorm(Ex[t], sigma)
		}
	}
}
}
\end{lstlisting}

Could take the posterior of random year effects and look at what explains that variability. Plot those posteriors against covariates and then add those into the model.

\subsubsection{Model 3: Random block effect}

\begin{lstlisting}
model{

mu ~ dnorm (0, 0.001)
sigma ~ dgamma(0.001,0.001)

for(b in 1:nb) { alpha.b[t] ~ dnorm (0, tau.b) }
tau.b ~ dgamma(0.001,0.001)

for(b in 1:nb){
	## process model
	Ex[b] <- mu + alpha.b[b]
	for(t in 1:nt){
		for(i in 1:nrep){
		x[t,b,i] ~ dnorm(Ex[b], sigma)
		}
	}
}
}

\end{lstlisting}

\subsubsection{Model 4: Random block and time effect}

\begin{lstlisting}
model{

mu ~ dnorm (0, 0.001)
sigma ~ dgamma(0.001,0.001)

for(t in 1:nt) { alpha.t[t] ~ dnorm (0, tau.t) }
for(b in 1:nb) { alpha.b[t] ~ dnorm (0, tau.b) }
tau.b ~ dgamma(0.001,0.001)
tau.t ~ dgamma(0.001,0.001)

for(t in 1:nt){
	## process model
	for(b in 1:nb){
	Ex[t,b] <- mu + alpha.b[b] + alpha.t[t]
		for(i in 1:nrep){
		x[t,b,i] ~ dnorm(Ex[t,b], sigma)
		}
	}
}
}

\end{lstlisting}

It's really challenging to identify the number of parameters in a hierarchical model. The number of parameters is ill-defined by the structure of the model; instead the number of parameters in the model depends on what data it's fit to. 

\subsection{Nonlinear hierarchical models}

These are possible but often take a bit more thought for deciding which parameters are considered random or fixed. One issue is that setting all parameters to random can often result in unidentifiability. Inclusion of covariates is also challenging

\subsection{Prediction}

Hierarchical models facilitate prediction about an unobserved species, site, years, etc. Formally, out-of-sample predictions integrate over random effects and are more uncertain than in-sample. The posterior for a new species, site, year could be updated with a relatively small number of observations. Can improve predictions because they have this prior already from the HM model.

\section{Expert elicitation}

Presenter: Melissa Kenney and Mike Dietze

Develop probability distributions on priors. 
Develop probability distributions on likelihoods.
Elicit structural aspects of the model.

The literature is often full of uninformative priors. For those who have prior knowledge about phenomena that we are interested in and have information about, expert elicitation helps construct meaningful priors. Expert elicitation is particularly useful when working in data limited situations. 

An example where expert elicitation and model based forecasting have been combined is weather forecasting. For example, in the 1960s weather forecasts didn't do so well. But local weather forecasters did. In some cases, experts actually outperformed models initially. The experts were an important part of the forecast process and helped inform models.

Iterative: repeated updating of forecasts as new data become available. All of this data helped experts learn how they were doing with their forecasts and to improve their own forecasts over time. Models had the same benefits of making forecasts and seeing their forecasts improve over time.

Thinking about formalized expert judgment and model forecasts combine is a key way of improving ecological forecasts.

Problems that occur when constructing an expert elicitation. It's not trivial. We can think about our own judgment but if you are formally incorporating probabilistic elicitations, it's hard! This requires a lot of forethought and collaboration with social scientists. 

Grainger-Morgan PNAS 2014 \cite{morgan2014} identifies a few problems/issues. First, you need to focus on a topic where an expert can make an informative, predictive judgement. This isn't always the case. Need to set up the problem so someone can answer a probabilistic judgment. Second, terms like "likely" "about normal" are imprecise and instead need to focus on precise quantitative probabilities. Third, when working with experts are often over-confident and may construct over-confident probabilities due to mental shortcuts/heuristics. Fourth, want to get the right experts. You want to ask the right people. Capture a diversity of expert opinion. Fifth, you want to spend time on developing, revising, and implementing your elicitation protocol. Sixth, the situation needs to be realistic (e.g. give experts space/time to give answers). Finally, you would need to consider whether you want to aggregate the expert opinions into a single probability distribution or keep them separate.

Expert elicitation forces you to think about your priors. 

Mike Dietze also not a fan of uniform priors.

Elicitation often happens with Cumulative Distribution Functions. Also, elicitation also works from outside in. For example, by guessing min and max rather than mean!

\section{State-space}

Presenter: Shannon LaDeau John Foster'

Broad framework for thinking about models is signal and noise. Confidence intervals get tighter by adding in more data. Predictive intervals - you add in un certainty. Width is based on observation error and how well you characterized the model. 

When we're forecasting we often think beyond the range of the data that we have. The confidence and predictive intervals get bigger. Many of the issues with uncertainty come into play here: we're outside of the range of our data, there is uncertainty in the covariates. 

Ecologists have not had a lot of success in forecasting 5-10 years out. 

\subsection{When to use state space models}

State-space models are useful when latent variables are connected in time or space. State-space models are also dynamic models, where the current state depends on the past state. State in this vocabulary is the variable of interest. 

\begin{align}
X_{t+1} = X_t
\end{align}

A Bayesian state-space model has the process model and data model. 

\begin{align}
X_{t} = f(X_{t-1})
\end{align}

Missing

State-space models are flexible frameworks for understanding how states of interest change through time. These models are quite general:

\begin{itemize}
\item ....
\end{itemize}

\subsection{Random walk state space model}

\begin{align}
X_{t} = f(X_{t-1}) + \epsilon_t \\
Y_t = g(X_t) + \omega_t \
\end{align}

...

Graphically, this can be broken into a process model that links $X_{t-1}, X_t, X_{t+1}$ that are dependent on each other. There are parameters that are connected to all of the Xs. The Ys are conditionally independent on the Xs. 

\begin{lstlisting}
RandomWalk = model{

## data model

## process model

## parameters

\end{lstlisting}

The further you get from data, the more uncertainty there is. Shannon presents an example with flu data. 

\subsection{Dynamic linear state space model}

\begin{align}
X_{t} = f(X_{t-1}) + \beta_0 +\beta_1 X + \epsilon_t \\
Y_t = g(X_t) + \omega_t \
\end{align}

Parts of the model are: 

X = latent time series
Y = observed dataq
$\beta$ = parameters in linear model
$\epsilon$ = process error
$\omega$ = observation error

\subsection{Nonlinear state space model: capture-recapture}

One example of this is mark-recapture data. Individuals are capture, marked, recaptured...

Data in this case look like this. An individual record consists of $Y_i = [1,0,1,0,0]$. This is compatible with several true states:

$$Z_i = [1,1,1,0,0]$$
$$Z_i = [1,1,1,1,0]$$
$$Z_i = [1,1,1,1,1]$$

The different possibilities are that it died, that it survived one more sampling period, or survived the entire sampling period. We don't know the exact time of death. We also do know that we didn't capture it in the second census. In the example that was put up, the parameter model also has variance that changes throughout time. 

Survival is modeled with a Bernoulli. The process model has four possibilities (survival porbabilities):

\begin{align}
P(X_t = 1 | X_{t-1} = 1) = s_t \\
P(X_t = 1 | X_{t-1} = ) = 0 \\
P(X_t = 0 | X_{t-1} = 1) = 1- s_t \\
P(X_t = 0 | X_{t-1} = 1) = 1
\end{align}

The observation model has four possibilities (detection probabilities) [missing]

\begin{align}
P(X_t = 1 | X_{t-1} = 1) =  \\
P(X_t = 1 | X_{t-1} = ) = \\
P(X_t = 0 | X_{t-1} = 1) =  \\
P(X_t = 0 | X_{t-1} = 1) = 
\end{align}

There's also a parameter component.

\subsection{Assessment}

\begin{enumerate}
\item 1. The ?latent state? is our variable of interest, which we infer from observations, but cannot measure perfectly. TRUE. The goal here was to make sure that we understand what the latent state is. 
\item Using state-space models, we separate observation error and process error. TRUE. This is one of the strengths of state-space models. It's also challenging because there's a tradeoff. If there's any way of constraining observation error, you can get a better estimate of process error. The state-space framework focuses on these two sources of error.
\item State-space temporal models are also called dynamic models or hidden Markov models. TRUE. Dynamic models refer to the time series component. State-space models and hidden Markov models are interchangeabley used. You can write dynamic models that are not state-space models. You can also write Markov models that are not state-space models. It's the hidden component that connects it to latency.
\item State-space models are limited by common statistical assumptions, such as independent observations or Normally-distributed data. FALSE
\item Random walk models use the current state, plus error, to make forecasts. TRUE
\item A random walk model is often a reasonable null model. TRUE
\item Error terms must be static over time. FALSE
\end{enumerate}

\section{Dynamic models}

Presenter: Mike Dietze

Dynamic models explain change over time. These are models that describe how the state of the system is changing over time: the future state is a function of the current state (and lags). Dynamic models force us to focus on the process that causes change. These models behave differently from regression. 

Dynamic models are different from regression models, and behave differently.

\subsection{Examples}

Persistence/random walk: $ X_{t+1} = X_{t} $

Exponential growth/decay: $ X_{t+1} = \rho X_{t} $

[name?]: $ X_{t+1} = \beta_0 + \rho X_{t} $ In this model the equilibrium is $\beta_0/(1-\rho)$. $\rho$ controls the rate of approach to equilibrium. This model is a classic first order autoregressive model (??). A $\rho$ close to 1 would imply high autocorrelation, whereas close to 0 would imply independence.

Quadratic: $ X_{t+1} = \rho X_{t} + \beta_1 X_{t}^2 $ This is equivalent to the logistic growth model. These are connected if $\rho = 1+r $ and $\beta_1 = -r/K $.

Compatibility of hypothesis and forecasting. All of these different models are hypotheses about a dynamic system, and if you make forecasts with multiple models they are alternative hypotheses about the system works.

Environmental covariate: $ X_{t+1} = X_{t} + \beta_1 Z_t $ This is a population growth model with an environmental driver. The effect this has is to add some variability around the mean persistence.

Something that happens a lot with shorter timescales, there are often environmental drivers that have seasonal or diurnal cycles as well as data that have cycles. While a seasonally varying Z may drive a seasonally varying X, they may not be in sync. There may be a lag/shift out of phase.

These can become more and more complicated. So, for example, you might consider the change in Z: Z(t+1)-Z(t) as the covariate. 

What if, instead, the temperature is setting the carrying capacity. In this case, the carrying capacity becomes a linear function of the driver variable. 

Can also consider an AR[2] autoregressive. 

In addition to including current state into the system, you can also include previous states and that have memory (AR(2)). These can also help capture trends in systems. 

\subsection{Building a process based model}

The best place to start is a classic box and arrow diagram. Boxes represent states, solid arrows represent fluxes, and dashed arrows represent influences. This can help you develop an understanding of your system without worrying about the specifics of equations. This can then help you develop the equations.

Developing the balance equations comes next. This helps you conserve mass. You can also include terms that are functions of covariates.

The next step would be to develop your flux equations. For example, what describes the transfer processes.

\subsection{When are dynamic models appropriate for forecasts?}

The distinction between a statistical and dynamic model is a subtle one. From the perspective of forecasting, there's not a big difference between phenomenological versus process-based models. Both use the same statistical frameworks. The process models are slower to fit because there are more parts. (???)

If we want to predict tomorrow, knowing what's happening today/knowing where you are right now really should be one of the covariates for where you are going to be tomorrow. Step back and think about how you're treating the model because you are describing dynamics rather than fitting a curve through an x-y scatterplot.

You can also take a process-free and equation-free approach to predicting the future. Some of this comes out in machine learning approaches (see lecture this afternoon). Maybe return to this for Thursday afternoon roundtable. 

\section{Machine learning}

Presenter: Ethan Deyle

Nonlinear forecasting can be traced back to a 1990 paper by Sugihara and May (\cite{sugihara1990}). They used a nearest-neighbor approach to forecasting multivariate data. Also Edward Lorenz's work in climate and "knn nearest-neighbor" pattern recognition.

Terminology: machine learning, non-parametric, model-free, non-structural. What we're really looking at is prediction without parameters. As a contrast, Dietze (\cite{dietze2017a}).

$$Y_{t+1} = f(Y_t, X_t | \bar{\theta} ) + \epsilon_t$$

All of these approaches come back to universal function approximators. This includes nearest-neighbor forecasting, generalizations of regression, ...

When do you want to use non-parametric approaches? When there might be no good parametric equations, unknown or unmeasured variables, unmanageable parametric complexity. Another case might be when you're trying to integrate big data stream or if there's non-traditional information.

Google Flu was a project that thought it would be possible to use search data to predict flue trends. \cite{lazer2014} diagnose problems with the Google Flu approach.

It's hard to apply information theoretic (AIC/BIC) measures of data and it can be easy to include arbitrarily good fits to the data. 

This makes it important to do cross-validation. There's a model training data set, and the forecast skill is evaluated on a separate set. Sometimes this is done by splitting the data into parts. When the data are really limited, you might do "leave-one-out" (LOO).

\subsection{Empirical dynamic modeling}

Takens' Theorem...

The underlying idea is to identify the geometry of the data. Video talks about taking the lag of a time series and trying to recover the full extent of the data ??? Similar to the idea that you're going to rewrite a system of ODEs equations.... Natural relationship between time lags and derivatives.

Relationship between forecasting and causality. Made by Clive Granger (\cite{granger1969}): If Y is causing X, then our forecast should be better if we include Y in our forecast than if we exclude it. This depends on separability of causes and effects that isn't applicable to a lot of realistic ecosystems. 

Convergent cross-mapping is relevant...

\subsection{Case study: recruitment in marine fisheries}

Parametric approaches use age-structured models. These models assume constant mortality across adult age classes and recruitment that needs to be forecast. There are a few parametric models that are used to model recruitment: Ricker, Beverton-Holt, Schnute...

Poor fits are common! One hypothesis is that the measurements of recruitment are too uncertain or too incomplete. Another is that recruitment is driven by stochastic environmental factors. Another is that the models of recruitment are wrong/incomplete. 

The video shows an example of a deterministic model that seems to show process/observation error but in fact is a product of an unmeasured 

Pierre et al. 2018, Much et al. 2018, Giron-Nava et al.

All use variations of empirical dynamic modeling to investigate recruitment prediction/the stock recruitment relationship.

\subsection{Other approaches to nonparametric estimation}

cognub.com/index.php/cognitive-platform.

Symbolic regression: predicting time-to-flowering in chickpeas incorporating climate drivers and genetic variation.

\section{PROACT: Structured decision making}

Presenter: Melissa Kenney

\subsection{Decision support}

Major components of a decisions are: uncertain scientific information and forecasts, individual or stakeholder values, and constraints (environmental, social, or regulatory).

When we talk about decision support it's not just forecasts, but also processes to allow scientists and decision makers to work together. It's important to figure out ways of co-producing ecological forecasts so that scientists and stakeholders are working together to produce useful information. It's not just a model where we push out information and hope it's useful. But we're also not trying to develop something that's useful to only a narrow set of people.

Work with stakeholders to develop forecasts related to decisions that they care about. How do we make projections that are related to specific interventions. Ecological forecasts are a virtual laboratory. This is what Hollings talks about on his book on adaptive management. 

Models are an alternative to unethical interventions. Interesting to think about this as a 'benefit of models'.

How do we break down models? One acronym is Problem Objective Alternatives Consequences Trade-offs: PROACT. This is discussed in the decision support chapter.

This is not a linear process. Ideally you would be improving your management actions over time. 

Structured decision making is described in book by Robin Gregory (Structured decision making).

\subsubsection{Problems and objectives}

The first step is making sure that we have a problem that's appropriate for a decision context. There are multiple stakeholders that need to be part of the conversation. Understanding those perspectives and identifying a shared problem is critical. This is the first step to getting a shared solution.

What are your objectives? These are your goals. While we might care about many environmental goals, there are also social goals/objectives that must be part of the conversation. For example, cost, social equity, robust participatory process. Those are just as important and one way to consider them is to make sure all relevant stakeholders are heard.

If you're missing objectives, your model is likely incomplete. It doesn't mean it is not informative but you may not have all important information.

Some of the criteria to consider: complete, concise, controllable, understandable, preferentially independent (don't need to know about cost to know about maximizing ecological enhancement).

Exercise: What are the decision objectives when making a decision about planning a trip to Boston (specifically the transit). First, propose 3-4 decision objectives and specify the desired direction of change (max/min).

Performance measures are linked to objectives. We need to have information that predicts particular performance measures (PMs). For a given objective, you can have 1 or more PMs. For example, if you are interested in water quality, you could look at various aspects of water chemistry.

A big design challenge is how you represent the performance measure.

Let's take population growth of a species. Types of measures: natural measures (abundance), proxy (area of habitat), or a constructed scale (this could be performance measure specific). The last can translate latent concepts into concrete concepts. 

Time has natural performance measures. Productivity have proxy or scale measures: number of people, wifi connectivity, 5 star rating for travel.

Use of threshholds can also be helpful for dealing with probabilistic outcomes

\subsubsection{Alternatives}

What are your options, what decisions can you make? This is one of the most difficult pieces because if we start with what we care about it may be easier to come up with more options to consider.

This idea is called "value-focused thinking". Starting by thinking about what you care about will help you think more creatively about what you want to accomplish.

This is a place where scientists can be transparent and defensible. The timeframe is really important as well.

Will also help with identifying boundary conditions.

The alternatives have to be complete. In the travel example, you have to consider transit time door-to-door. This is the idea of making apples-to-apples comparisons.

\subsubsection{Consequences}

This is where forecasts come into play. Forecasts can be projections that are the result of combining boundary conditions and scenarios.

An important thing when combining forecasts with decisions is the time horizon. A mismatch means that the information that's produced might not be useful.

This is where performance measures come into play. Want to think about designing model output that links to specific objectives that we care about.

One of the reasons that ecologists and decision makers have to think hand-in-hand, is that if they don't then we end up with forecasts that are not useful. Alternatively, it might not be possible to provide the kind of information that a decision maker is asking for.

\subsubsection{Trade-offs}

How do we make choices between different alternatives? This is challenging because we are often measuring multiple things, have many different choices, care about different things, and have uncertainty. 

A decision scientist or operations research expert can help apply these methods in a robust way.

Three major tradeoff methods: dominance, robust, multi-attribute criteria. Dominance: there's never a better choice. Robust: likely options that may allow for flexibility. Multi-attribute criteria methods: comparing across a range of different objectives.

We can often combine objectives with a weighted utility function. This is important when we have multiple stakeholders.

There are rigorous ways to assess these utilities by doing preference elicitation: mapping between desirability and utility. Can also construct weights.

It's often not our goal to make the final judgment, that's the decision makers goal.

Dominated means that it is equal or worse on all measures. Below, there is a better option than train on all objectives (bus on cost, plane on time, productivity in bus, disease risk for drive).

Increases in uncertainty translate into increases in risk.

Working in order is important - don't jump ahead to measurement! This gives you an objective/structured way of assessing decisions/actions. Can get people agreeing to actions but for different reasons.

\subsubsection{Sample decision matrix}

Here's an example that we worked with in the class as a contingency table:

\begin{table}[h!]
\begin{tabular}{|l|l|l|l|l|l|}
\hline
Objective & Measure & Airplane & Drive & Train & Bus \\ \hline
minimize cost & dollars  & 200  & 100 & 45 & 5  \\ \hline
minimize time & hours & 2 & 4 & 4 & 4 \\ \hline
maximize productivity during transit & wifi quality & 3  & 1 & 4 & 4 \\ \hline
minimize disease risk & people in vehicle  & 100  & 1  & 100  & 60 \\ \hline
\end{tabular}
\end{table}

\section{Propagating uncertainty}

How does a change in X translate into a change in Y? [$\delta x -> \delta y$]

\subsection{Sensitivity analysis}

One numerical method is: One at a time perturbation. Modify individual parameters and look at the response. Something that has changed a lot is that previously sensitivity analyses were done at arbitrary e.g. 10\% but if you have an uncertainty then you can do sensitivity at 1, 2, 3 standard deviations or quantiles for non normal distributions.

Monte Carlo sensitivity analysis is one of the global sensitivity analyses. Here, these allow you to look at the full shape of the sensitivity. Fitting a regression slope through cloud of points will tell you something about the average sensitivity. The other one is the partial R2 associated with input. Partial R2 tells you how much is it contributing to the overall variability.

The main tradeoff is computational: how well can you characterize the global sensitivity versus how much it will cost you computationally.

\subsection{Uncertainty propagation}

Transforming uncertainty into the future is more generally a problem of transforming functions.

An analytic approach gives us an equation. A numeric approach is often computationally demanding.

What do we want to get out? Maybe the full distribution? Or just the statistical moments?

This is table 11.1 in the textbook: 

\begin{table}[h!]
\begin{tabular}{l|l|l|}
 & Distribution & Moments  \\ \hline
Analytic & Variable transform  & Analytical moments or Taylor Series    \\ \hline
Numeric & Monte Carlo & Ensemble  \\ \hline
\end{tabular}
\end{table}

Analytical methods for variable transformation are the gold standard. Fairly non-trivial. These are calculus-demanding!

Analytical moments. There are clear rules for these. For example: $Var(aX) = a^2 Var(X)$. Algebra of means is quite intuitive but the algebra of variances is a bit trickier. These methods only work out if you are using linear combinations of random variables. Non-linear models become more complicated and analytical moments don't provide an exact solution.

Let's see what this is doing:

\begin{align} y = m \theta + b \\
m = \delta y / \delta \theta \\
Var[y] = Var[m \theta + b ] \\
Var[y] = m^2 Var[\theta] \\
Var[y] = (\delta y / \delta \theta)^2 Var[ \theta ]
\end{align}

These equations emphasize that the uncertainty in a prediction is determined by (1) the uncertainty in the model's inputs ($Var(\theta)$) and (2) the sensitivity of the prediction to those inputs ($\delta y / \delta \theta$). So you can be misled by focusing on parameters that are really sensitive but that you know very well. Instead, uncertainty can be determined by what you don't know much about!

What about non-linear models? For one, you can use linear approximations (estimate with a straight line...)

\begin{align} y = f(x \mid \theta) \\
Var[y] \sim (\delta y / \delta \theta)^2 Var[ \theta ]
\end{align}

This is related to using a Taylor series decomposition. Eg. for one parameter

\begin{align}
Var[ f(x \mid \theta)] \sim Var[ f(x \mid \bar{\theta}) + \frac{ \frac{df}{d\theta} ( x \mid \bar{\theta} )}{1!} ( \theta - \bar{\theta})] + ... \\
var[f(x)] \sim  (\frac{df}{d\theta_i} )^2 var[\theta]
\end{align}

For multiple parameters this becomes more complicated. The covariances among your parameters becomes really important!! The full joint posterior of your distributions are important. You have to sample for the joint distribution. It's that correlation that gives a classic hourglass shape in confidence intervals.

\begin{align}
Var[ f(x \mid \theta)] \Sigma Var[ f(x \mid \bar{\theta}) + \frac{ \frac{df}{d\theta} ( x \mid \bar{\theta} )}{1!} ( \theta - \bar{\theta})] + ... \\
var[f(x)] \sim \Sigma (\frac{df}{d\theta_i} )^2 var[\theta_i] + \Sigma_{i\neq j} ( \frac{df}{d\theta_i} )(\frac{df}{d\theta_j} ) cov(\theta_i, \theta_j)
\end{align}

Numerical ways of propagating uncertainty. We can take samples of a distribution and transform those samples through a function to another distribution. 

Jensen's inequality states that the mean of a function is not equal to plugging the mean inputs into that function. Because of this, you can not transform mean or standard deviations. What you can transform is samples of a distribution (this gets around Jensen's inequality).

Monte Carlo Simulation --> distributions. You take these samples from the distributions, run models, and then save the results. What you're sampling here is that you want to sample rows from your posterior (to properly capture covariances across parameters). There's a simulation that shows that when the number of. For each of the lines, you're getting an intercept and variance from the posterior distribution. Then you're drawing a line and repeating this process is what results in the classic hourglass characterization of uncertainty. To get an interval is to apply quantile function across. 

Ensemble analysis --> moments. People usually are talking about a smaller number of samples for this. Ultimately, the goal is to fit PDF to the results and using the PDF for intervals. This is in part because you need smaller N to estimate the moments. This is in part a tradeoff because you are assuming a shape by doing so. Generally, for approximating a distribution you want an effective sample size of ~5000. If you are trying to get a stable estimate of (eg.) mean you can do so with tens or hundreds. For example, this is weather forecasts! 

Choice between the two: if you have a computationally efficient model you can run the full Monte Carlo method.

\subsection{Uncertainty analysis}

Idea is to estimate the contribution of each input to output. Things can be important because they are uncertain or sensitive.

You can have high predictive uncertainty with high sensitivity/high uncertainty, high sensitivity/low uncertainty, or low sensitivity/high uncertainty.

Variance decomposition with switchgrass yield. One way of viewing this is: plotting CV for parameters, plotting elasticity for parameters. So you can see that the partial variance of parameters is driven both by CV and elasticity. (e.g. \cite{raczka2018}). These can provide information for model-data feedbacks.

\subsection{Optimal design}


\section{Data assimilation: analytical methods}

Presenter: Mike Dietze

Iteratively updating models as new data becomes available. This is part of the forecast-analysis cycle.

\subsection{Analysis Problem}

Before we observe how the future plays out, what is the best estimate for the future state of the system $X_{t+1}$? This ends up being the forecast, $P(X_{t+1})$

Once we make (imperfect) observations of the system, $Y_t$, what's our best estimate of $X_t$?

$$P(X_{t+1}) = P(Y_{t+1}) $$

$$P(X_{t+1} \mid Y_{t+1}) \prop P(Y_{t+1} | X_{t+1}) P(X_{t+1}) $$

This is equivalent to the posterior, which is proportional to the likelihood times the prior (the forecast). This means that the analysis problem itself is Bayes theorem. (this is the key!!)


\section{Ensemble data assimilation}

\section{Social science}

\section{Model assessment}

\section{Forecast infrastructure/FLARE case study}


%%% LIST
\begin{itemize}
\item 

\item 

\item 

\end{itemize}

%%% MATH
\begin{align}
\dot{x_1} & = u(t) x_1 \nonumber \\
\dot{x_2} & = (1-u(t)) x_1 
\end{align}

%%% MATRIX
\begin{equation}
A_{11}=
  \begin{bmatrix}
    -1		& 0 	 	& \ldots 	& 0  	 \\
    0 	 	& -1  	& \ddots 	& \vdots  \\
    \vdots   & \ddots 	& -1		& 0  \\
    0 		& \ldots 	& 0 		& -1  
  \end{bmatrix}_{ t \times t}
\end{equation}

%%% CODE CHUNK 
\begin{lstlisting}
## Function that computes values of derivatives in the ODE system
## Takes the time step, system of differential equations, parameters
derivs = numeric(2); 
control <- function(times0,y,parms,f1,...) {
  
  # x1 and x2 are the two entries in y (ode)
  x1=y[1]; 
  
  # values calculated by the interpolated function at different time points
  u <- f1(times0);
  
  derivs = c(u*x1,(1-u)*x1) 
  return(list(derivs));
}
## Compiles the function control()
control=cmpfun(control); 
\end{lstlisting}

\clearpage
\nocite{*}

\bibliographystyle{/Users/gregor/Dropbox/bibliography/styleFiles/ecology} 
\bibliography{/Users/gregor/Dropbox/bibliography/forecasting-course}

\end{document}